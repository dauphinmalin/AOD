\documentclass[a4paper, 10pt, french]{article}
% Préambule; packages qui peuvent être utiles
   \RequirePackage[T1]{fontenc}        % Ce package pourrit les pdf...
   \RequirePackage{babel,indentfirst}  % Pour les césures correctes,
                                       % et pour indenter au début de chaque paragraphe
   \RequirePackage[utf8]{inputenc}   % Pour pouvoir utiliser directement les accents
                                     % et autres caractères français
   % \RequirePackage{lmodern,tgpagella} % Police de caractères
   \textwidth 17cm \textheight 25cm \oddsidemargin -0.24cm % Définition taille de la page
   \evensidemargin -1.24cm \topskip 0cm \headheight -1.5cm % Définition des marges
   \RequirePackage{latexsym}                  % Symboles
   \RequirePackage{amsmath}                   % Symboles mathématiques
   \RequirePackage{tikz}   % Pour faire des schémas
   \RequirePackage{graphicx} % Pour inclure des images
   \RequirePackage{listings} % pour mettre des listings
% Fin Préambule; package qui peuvent être utiles

\title{Rapport de TP 4MMAOD : Génération d'ABR optimal}
\author{
NOM Prénom étudiant$_1$ (groupe étudiant$_1$) 
\\ NOM Prénom étudiant$_2$ (groupe étudiant$_2$) 
}

\begin{document}

\maketitle

%%%%%%%%%%%%%%%%%%%%%%%%%%%%%%%%%%%%%%%%%%%%%%
\paragraph{\em Préambule}
{\em \begin{itemize} 
   \item Compléter ce patron de rapport en supprimant toutes les phrases en italique\,: elles ne doivent pas apparaître dans le rapport pdf.
   \item Il sera attribué {\bf 1 point} pour la qualité globale du rapport\,: présentation, concision et clarté de l'argumentation.
\end{itemize}
}

%%%%%%%%%%%%%%%%%%%%%%%%%%%%%%%%%%%%%%%%%%%%%%
\section{Principe de notre  programme (1 point)}
{\em Mettre ici une explication brève du principe de votre programme en  précisant la méthode implantée (récursive, itérative) et les
choix effectués (notamment pour l'ordonnancement des instructions).
} 

%%%%%%%%%%%%%%%%%%%%%%%%%%%%%%%%%%%%%%%%%%%%%%
\section{Analyse du coût théorique (2 points)}
{\em Donner ici l'analyse du coût théorique de votre programme en fonction du nombre $n$ d'éléments dans le dictionnaire.
 Pour chaque coût, donner la formule qui le caractérise (en justifiant brièvement pourquoi cette formule correspond à votre programme), 
 puis l'ordre du coût en fonction de $n$ en notation $\Theta$ de préférence, sinon $O$.}

  \subsection{Nombre  d'opérations en pire cas\,: }
    \paragraph{Justification\,: }
    {\em La justification peut être par exemple de la forme: \\ 
       "Le programme itératif contient les boucles $k_1=...$, $k_2= ...$ etc correspondant à la somme 
      $$T(n_1, n_2, c_1, c_2) = \sum_{k_1=...}^{...} ... \sum ... + \sum_{i=...}^{...} ...$$ 
      somme que nous avons calculée (ou majorée) par la technique de  ... " \\
      ou  encore\,:  \\
      "les appels récursifs du programme permettent de modéliser son coût par le système d'équations aux récurrences 
      $$T(k_1, k_2) = ...  \mbox{~avec~les~conditions~initiales~....~} $$
      Le coût indiqué est obtenu en résolvant ce système par la méthode de  .... "
    } 
  \subsection{Place mémoire requise\,: }
    \paragraph{Justification\,: }

  \subsection{Nombre de défauts de cache sur le modèle CO\,: }
    \paragraph{Justification\,: }


%%%%%%%%%%%%%%%%%%%%%%%%%%%%%%%%%%%%%%%%%%%%%%
\section{Compte rendu d'expérimentation (2 points)}
  \subsection{Conditions expérimentaless}
     {\em Décrire les conditions permettant la reproductibilité des mesures: on demande la description
      de la machine et la méthode utilisée pour mesurer le temps.
     }

    \subsubsection{Description synthétique de la machine\,:} 
      {\em indiquer ici le  processeur et sa fréquence, la mémoire, le système d'exploitation. 
       Préciser aussi si la machine était monopolisée pour un test, ou notamment si 
       d'autres processus ou utilisateurs étaient en cours d'exécution. 
      } 

    \subsubsection{Méthode utilisée pour les mesures de temps\,: } 
      {\em préciser ici  comment les mesures de temps ont été effectuées (fonction appelée) et l'unité de temps; en particulier, 
       préciser comment les 5 exécutions pour chaque test ont été faites (par exemple si le même test est fait 5 fois de suite, ou si les tests sont alternés entre
       les mesures, ou exécutés en concurrence etc). 
      }

  \subsection{Mesures expérimentales}
    {\em Compléter le tableau suivant par les temps d'exécution mesurés pour chacun des 6 benchmarks imposés
              (temps minimum, maximum et moyen sur 5 exécutions)
    }

    \begin{figure}[h]
      \begin{center}
        \begin{tabular}{|l||r||r|r|r||}
          \hline
          \hline
            & coût         & temps     & temps   & temps \\
            & du patch     & min       & max     & moyen \\
          \hline
          \hline
            benchmark1 &      &     &     &     \\
          \hline
            benchmark2 &      &     &     &     \\
          \hline
            benchmark3 &      &     &     &     \\
          \hline
            benchmark4 &      &     &     &     \\
          \hline
            benchmark5 &      &     &     &     \\
          \hline
            benchmark6 &      &     &     &     \\
          \hline
          \hline
        \end{tabular}
        \caption{Mesures des temps minimum, maximum et moyen de 5 exécutions pour les 6 benchmarks.}
        \label{table-temps}
      \end{center}
    \end{figure}

\subsection{Analyse des résultats expérimentaux}
{\em Donner  une réponse justifiée  à la question\,: 
              les  temps mesurés correspondent ils  à votre analyse théorique (nombre d’opérations et défauts de cache) ?
}

%%%%%%%%%%%%%%%%%%%%%%%%%%%%%%%%%%%%%%%%%%%%%%
\section{Question bonus\,: programme {\tt mystere.c}(2 points)}
\subsection{Que fait le programme mystère et dans quel but? (1.5 point)}
{\em Dire brièvement ce que fait le programme mystere et quel est l'impact lors de l'exécution 
(i.e. lors de recherches avec le dictionnaire) de ce post-traitement.}

\subsection{Qu'en pensez-vous? (0.5 point) } 
{\em Répondre à l'argumentation en justifiant : soit que le programme mystere est (presque) optimal (justifier les hypothèses) ; soit qu'il n'est pas optimal en 
justifiant comment faire encore mieux.
}

\end{document}
%% Fin mise au format

